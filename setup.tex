
\usepackage[top=12ex, bottom=12ex, left=9ex, right=9ex,showframe=false]{geometry}
\linespread{1.8}
\usepackage{fancyhdr}
\usepackage{ctex}
\pagestyle{fancy}
\lhead{ \fancyplain{}{Yuhao Yang} }
\rhead{ \fancyplain{}{\leftmark} }
\cfoot{ \fancyplain{}{\thepage} }
\usepackage[unicode]{hyperref}
\usepackage{amsmath}
\usepackage{amsfonts}
\usepackage{latexsym}
\usepackage{pifont}
\usepackage{fourier}
\usepackage{wasysym}
\usepackage{stmaryrd}
\usepackage[dvipsnames,svgnames]{xcolor}
\usepackage{extarrows}
\usepackage{chemarrow}
\usepackage{graphicx}
\usepackage{tikz}
\usepackage{pgfplots}
\usetikzlibrary{arrows,positioning,calc,fadings,shapes,decorations.markings}
\usepackage{array}
\usepackage{anyfontsize}
\usepackage{circuitikz}
\usepackage{multirow}
%\usepackage{dingbat}   % special symbols
\usepackage{indentfirst}
\usepackage{paralist}
\usepackage{enumitem}
\usepackage{framed}
\usepackage{tcolorbox}
\tcbuselibrary{skins, breakable, theorems}
\usepackage{easyphys}
\usepackage{wrapfig}
\usepackage{caption}
\usepackage{titletoc}
\usepackage[explicit]{titlesec}
\usepackage{makeidx}
\usepackage{xfrac}




\newtcbtheorem{Question}{Question~(Science}%
  {enhanced, breakable,
    colback = white, colframe = gray!70, colbacktitle = gray!70,
    attach boxed title to top left = {yshift = -2mm, xshift = 5mm},
    boxed title style = {sharp corners},
   
    fonttitle = \sffamily\bfseries, separator sign = {).~}}{qst}

\titleformat{name=\section}[block]
{\begin{center}\begin{tikzpicture}}
{\draw[line width=4pt, Gray!70] (0,0) rectangle (12,3.2);
\node at (6,2.4) {\Huge\sc\bfseries\textcolor{purple}{Chapter \thesection}};}
{0pt}
{\node at (6,0.9) {\huge\filright\textcolor{purple}{#1}};}[\end{tikzpicture}\end{center}]

% table of contents sytle
\titlecontents{section}[9pc]
{\addvspace{10pt}%
	\begin{tikzpicture}[remember picture, overlay]%
	\draw[fill=Gray!70,draw=Gray!70] (-4,-0.1) rectangle (-0.8,0.5);
	\node at (-2.4,0.2){\color{white}\Large\sc\bfseries chapter\ \thecontentslabel};%
	\end{tikzpicture}\color{Gray}\hspace*{-10pt}\large\bfseries}%
{}
{}
{\;\titlerule\;\large\sc\bfseries Page \thecontentspage
	\begin{tikzpicture}[remember picture, overlay]
	\draw[fill=Gray!70,draw=Gray!70] (2pt,0) rectangle (6,0.1pt);
	\end{tikzpicture}}%
\titlecontents{subsection}[8.1pc]
{\addvspace{1pt}}
{\contentslabel[\thecontentslabel]{2.1pc}}{}
{\hfill\small \thecontentspage}[]
\titlecontents*{subsubsection}[8.1pc]
{\addvspace{-1pt}\small}{}{}
{\ - \small\thecontentspage}
[ /\ ][]


\renewcommand*\rmdefault{ppl}

\makeatletter


\renewcommand{\boxed}[1]{\textcolor{black}{%
\tikz[baseline={([yshift=-.72ex] current bounding box.center)}] \node [thick, rectangle, minimum width=1ex,rounded corners,fill=yellow!10, draw=orange] {\normalcolor\m@th$\displaystyle#1$};}}

\renewcommand\normalsize{%
	\@setfontsize\normalsize\@xpt\@xiipt
	\abovedisplayskip 0\p@ plus 5\p@ minus 3\p@
	\belowdisplayskip 0\p@ plus 5\p@ minus 3\p@
	\abovedisplayshortskip 0\p@ plus 5\p@ minus 3\p@
	\belowdisplayshortskip 0\p@ plus 5\p@ minus 3\p@
	\let\@listi\@listI}
\setlength{\@fptop}{0pt}

\renewcommand{\tableofcontents}{%
	\begin{tikzpicture}[remember picture, overlay]%
	\pgftext[right,x=13.5cm,y=0.2cm]{\color{Gray}\Huge\sc\bfseries \contentsname};%
	\draw[fill=Gray!70,draw=Gray!70] (11.42,-.75) rectangle (20,1);%
	\clip (11.42,-.75) rectangle (20,1);
	\pgftext[right,x=13.5cm,y=0.2cm]{\color{white}\Huge\sc\bfseries \contentsname};%
	\end{tikzpicture}%
	\vspace*{20\p@}%
	\@starttoc{toc}}
\makeatother

\DeclareGraphicsExtensions{.eps,.mps,.pdf,.jpg,.png}
\graphicspath{{figures/}}
\renewcommand\thefootnote{\textcolor{blue}{[\arabic{footnote}]}}
\hypersetup{
    colorlinks, linkcolor={purple}, citecolor={blue}, urlcolor={blue},
    pdftitle={Cambridge International A-Level Physics Course Notes},
    pdfauthor={Yuhao Yang},
    pdfsubject={A-Level Physics} }

\tikzset{>=stealth', pil/.style={ ->, thick, shorten <=2pt, shorten >=2pt,} }

\tikzset{note/.style={rectangle, rounded corners, minimum size=6mm, draw=black, fill=white, align=center,execute at begin node=\setlength{\baselineskip}{1.2em}}}
\tikzset{twoline/.style={align=center,execute at begin node=\setlength{\baselineskip}{1.2em}}}
\tikzset{twolinecap/.style={align=center,execute at begin node=\setlength{\baselineskip}{1.6em}}}

\newcounter{example}[section]
\newcounter{exercise}[section]
\newcounter{question}[section]
\renewcommand{\theexample}{\arabic{section}.\arabic{example}}
\renewcommand{\theexercise}{\arabic{section}.\arabic{exercise}}
\renewcommand{\thequestion}{\arabic{section}.\arabic{question}}

\newcommand{\question}[1]{%
	\refstepcounter{question}
	\noindent{\textcolor{magenta}{\textsf{\textbf{Question \thequestion }}} \hspace*{1pt} #1 %
}}
\newcommand{\exercise}[1]{%
	\refstepcounter{exercise}
	\noindent{\textsf{\textbf{Exercise \theexercise }} \hspace*{1pt} #1 %
}}

\newcommand{\cmt}{\noindent\hspace{-0.25em}\textcolor{Green}{\ding{226}} \hspace{0.2em}}
\newcommand{\sol}{\noindent\hspace{-0.12em}\textcolor{cyan}{\ding{45}} \hspace{0.2em}}
\newcommand{\solc}{\noindent\hspace{-0.12em}\textcolor{cyan}{\ding{45}} \hspace{0.2em} \vspace*{-\baselineskip}}

\newenvironment{ilight}
  {\centering
  	\vspace*{6pt}
  	\begin{tcolorbox}[colframe=Gray,colback=LightGrey!15]
  \setlength{\baselineskip}{\baselineskip}%
  }
  {\end{tcolorbox}\vspace*{-4pt}}


\newcommand{\keypoint}[1]{\textbf{\textcolor{red}{#1}}}

\setitemize{noitemsep,topsep=0pt,parsep=0pt,partopsep=0pt}
\numberwithin{equation}{section}
\numberwithin{figure}{section}
\everymath{\displaystyle}

\setlength{\parindent}{1.2em}

\newcolumntype{C}[1]{>{\centering\arraybackslash}p{#1}}
\newcolumntype{D}[1]{>{\centering\arraybackslash}m{#1}}

\makeindex